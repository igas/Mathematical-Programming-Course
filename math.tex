\documentclass[14pt,a5paper]{book}
\usepackage[T2A]{fontenc}
\usepackage[utf8]{inputenc}
\usepackage[english,russian]{babel}
\usepackage{amsmath}
\usepackage[a5paper]{geometry}
\geometry{left=2cm}
\geometry{right=1cm}
\geometry{top=2cm}
\geometry{bottom=2cm}
\begin{document}
  \begin{titlepage}
    \newgeometry{left=1cm,right=1cm,top=3cm,bottom=1cm}
    \begin{center}
      \Huge{Ю.~Я.~Зубарев}
    \end{center}
    \vfill
    \vfill
    \begin{center}
      \Large{\textbf{Математическое программирование\\
      Конспект лекций}}
    \end{center}
    \vfill
    \vfill
    \vfill
    \restoregeometry
  \end{titlepage}
  Математическое программирование - это методы принятия оптимальных решений.\par
  История развития методов математического программирования:
  \begin{enumerate}
    \item Системы автоматизации управления.\\Появились работы по оптимальным автоматическим системам.
    \item Методы, связанные с ракетной техникой.\\Решалась задача об оптимальной траектории ракет.
    \item Решение военных, экономических и административных задач.
    \item Логистика.
  \end{enumerate}\par
  Существовала и классическая теория оптимизации, но она не получила распространения, т.к. не учитывает ограничений.\par
  Критерий оптимальности - это формализованное правило, позволяющее производить сравнительную оценку различных вариантов (решений, процедур, операций...) и выбор наилучшей из них.\par
  Любая система (процесс, операция) характеризуется какими либо операциями.\par
\{СХЕМА!\}
  \begin{enumerate}
    \item Астрономические наблюдения.\\Судно - материальная точка.
    \item Волнения.\\Судно - линейная стохастическая модель.
    \item Разворот судна в порту.\\Нелинейная модель.
  \end{enumerate}\par
  Цель: свести решение задачи к типовым программным продуктам.\par
  Математическая формулировка задачи\par
  \(
  k(q_1, q_2\ldots q_n)\to min (max)
  \)
  \par
  3 показателя ограничений:\par
  \(
  k_{\rho}(q_1, q_2\ldots q_n)\geq k_{\rho_{min}}
  \)\par
  \(
  k_{\rho}(q_1, q_2\ldots q_n)\leq k_{\rho_{max}}
  \)\par
  \(
  k_{\rho}(q_1, q_2\ldots q_n) = k_{\rho_{0}}
  \)\par
  $\rho = 1, 2, \ldots m$ (обычно затраты) применяется редко\par
  Рассмотрим 1й случай:\par
  $k$ - линейные функции от параметров $q$. В этом случае мы рассматриваем задачи линейного программирования.\par
  $k$ - нелинейные функции. Тогда мы рассматриваем задачи нелинейного программирования. Размерность задачи нелинейного программирования 2\,--\,5 параметров.\par
  $q$ - целочисленные, тогда имеем дело с целочисленным программированием (линейным и нелинейным). Разбиение задачи на подзадачи.\par
  Стохастическое программирование - нам не известны точные значения $q$, но известны их вероятностные характеристики или интервалы изменения их значений.\par
  \chapter{Методы одномерной оптимизации}
  $$k(q)\to min(max)$$\par
  Унимодальная функция\par
  \{ГРАФИК\}\par
  \section{Методы с последовательным уменьшением интервала неопределенности}
  $\Delta_0 = q_{max} - q_{min}$ -- это интервал неопределенности.\par
  \newpage
  \begin{equation} \label{eq:someequation}
    \bar{q}^{\,r+1}=\bar{q}^{\,r}-\theta J^{\,-1}(q^{\,r})\nabla K(\bar{q}^{\,r})
  \end{equation}
В частном случае
\end{document}
